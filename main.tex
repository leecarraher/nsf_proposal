\documentclass[a4paper,10pt]{article}
\usepackage[utf8]{inputenc}
\usepackage{fullpage}
\usepackage{cite}
\usepackage{graphicx}
\usepackage{wrapfig}
\usepackage{url}
\usepackage[small]{caption}
\usepackage{subcaption}
\DeclareGraphicsExtensions{.pdf,.eps,.svg,.pgm, .png}
    % EXTREMELY COMMON LaTeX PACKAGES TO INCLUDE:
%    \usepackage{amsmath,amsthm, amsfonts,amssymb} % For AMS Beautification
    \usepackage{setspace} % For Single & Double Spacing Commands
    \usepackage[linktocpage,bookmarksopen,bookmarksnumbered,% For PDF navigation
		pdftitle={RP Hash Clustering},%   and URL hyperlinks
		pdfauthor={Department of Electronic and Computing Systems},%
		pdfsubject={UC },%
		pdfkeywords={UC}]{hyperref}




     %      \usepackage{caption3} % load caption package kernel first
    %\DeclareCaptionOption{parskip}[]{} % disable "parskip" caption option
%    \usepackage[small]{caption}
\usepackage{amsmath}
\usepackage{amsfonts}
\usepackage{amsthm}
\usepackage{amssymb}
\usepackage{float}
\usepackage{algorithm}
\usepackage{algcompatible}
%\usepackage{algorithm2e}
%\usepackage{algorithmic}



  \newtheorem{Theorem}{Theorem}[section]
    \newtheorem{Proposition}[Theorem]{Proposition}
    \newtheorem{Lemma}[Theorem]{Lemma}
    \newtheorem{Corollary}[Theorem]{Corollary}

        \newtheorem{Definition}[Theorem]{Definition}
        \newtheorem{Example}[Theorem]{Example}
        \newtheorem{Conjecture}[Theorem]{Conjecture}

        \newtheorem{Remark}[Theorem]{Remark}
    \newenvironment{Sketch}{\par\noindent{\sc Sketch of Proof}\quad}{\hfill\qed\par\smallskip}
    \newenvironment{Proof}{\par\noindent{\sc Proof}\quad}{\hfill\qed\par\smallskip}
    \newenvironment{ReuseTheorem}[2]{\par\vspace{12pt}\noindent{\bf #1~\ref{#2}$'$.}\it}{\par\vspace{12pt}}
    \newenvironment{RestateTheorem}[2]{\par\vspace{12pt}\noindent{\bf #1~\ref{#2}.}\it}{\par\vspace{12pt}}

\usepackage[section]{placeins}
%opening
\title{Random Projection Hashing for Scalable Big Data Clustering}
\author{Lee Carraher}
\begin{document}

\maketitle

\begin{abstract}



\end{abstract}
\section{Project Overview}
\emph{RPHASH} is a
novel method proposed for large scale distributed data clustering using locality sensitive
hash functions and random projection.
Clustering is an inherent problem in data analysis that faces a variety of issues when
applied to very large, distributed datasets.
The system designed herein will focus on data security, algorithmic and communication
scalability. The system 
will be compatible with commercial, cloud-based, distributed computing
resources(e.g.
Amazon EC2, Google Cloud, and similar offerings from IBM and Intel) as well as private
distributed processing systems. 
The proposed system will provide a straightforward interface
running on the popular MapReduce (\emph{MRv2})\cite{mapreduce} 
parallel processing system, to allow for arbitrary resource scalability.
The clustering system proposed will also have wide applicability to a variety of standard clustering
applications while focusing on a subset of clustering problems in the 
biological data analysis space.
Furthermore, due to new requirements involving the handling of health care data, as well as
the inherent privacy concerns of using cloud based services, data security is primary to RPHash'es
interprocess communication architecture.\\


\subsection{RPHash: A Secure, Scalable Clustering Algorithm}
The RPHash algorithm uses various recent techniques in data mining along with 
a new approach toward achieving algorthmic scalability. A large body of research 
has been applied to this field and in the past has focused on shared 
memory architectures with heavy data bandwidth requirements. Recent
architectures however, promoted by the commercial successes of Big Data and cloud based
on commodity hardware, have driven down system bandwidth, 
shifting related requirements for data clustering algorithms.
%%%%wonky sentence, need to fix, suggestions: two sentences maybe%%%%
RPHash is an algorithm designed with the priority of
minimizing interprocess communication requirements, in exchange for
some redundant computation. As with many parallel computations, and often computing
in general, memory access and synchronization bottlenecks often dominate the computational
complexity of an algorithm. For this reason, we propose redundancy of computation as
an acceptable trade-off, while hopefully achieving similar results of other communication
intensive clustering algorithms.\\

Our clustering system is intended for use by a variety of researchers in fields beyond
computing, and thus must require a minimal learning curve. One such way of
providing this, is by removing the technical hurdle of building and maintaining large
scale computing systems, and instead rely on commercially provided cloud computing
resources.
Furthermore, as the distributed processing architecture is
a commercially provided system, the companies commercial
ambitions will promote ongoing access to state-of-the-art computing
resources. Cloud computing provides a pay-per-use
computing resources that scale with the computational requirements 
without burdening researchers or their IT providers.\\

As RPHash is intended for cloud
deployment, data security should be a primary concern. 
Privacy concerns 
over attacks on de-identification, or de-anonymization\cite{deanon1}\cite{deanon2}
of thought to be anonymous data, such as  whole genome sequence data\cite{deident},
has prompted a presidential commission on the subject\cite{presidential}. The
results of which add new restrictions to the way researchers acquire and process data.
While attempting to mitigate communication restrictions, RPHash intrinsically provides some
security in the data it chooses to exchange. Namely, the randomly projected hash IDs, and
the aggregation of only the k-largest cardinality vectors sets.
%An important aspect of the 
%RPHash system is its focus on system independent
%data security for private clouds, connected via the internet. 
 Non-distributed data clustering requires the
entire dataset to 
reside on the processing system, while distributed methods often require
communication of  full data records between
nodes. In our approach, nearly-orthogonal random projection is a destructive
operation, that
provides privacy of the data being processed and communicated. While centroid
communication is 
by definition only transmitting the aggregate of data records, and in the case
of patient data, is not any single individual's information.\\

\subsection{Proposal Objectives}
The overall goal of the proposal is to create a secure, and scalable cloud based 
clustering system of high dimensional data such as genomic data. The following research
objectives will be addressed specifically:
\begin{enumerate}
\item \textbf{Develop the sequential RPHash Algorithm:}
 The first step in developing our parallel system is in developing the initial
 sequential version of the RPHash algorithm. From previous work, the difficult
 implementation of decoders and random projection methods have already been developed\cite{carraher}
 , the overall algorithm needs further implementation improvements to
 accurately match the sequential version of the proposed parallel algorithm.

  \item \textbf{Develop A distributed version of the proposed RPHash Algorithm:}
  As our current sequential algorithm for random projection hashing has shown both 
 accuracy and scalability promise, our next step is to extend the system to a parallel
architecture for which it was designed for. The parallel extension is a straightforward
application of the parallel prefix sum algorithm of Ladner\cite{Ladner} providing work
efficient parallel speed up.


  \item \textbf{ Deploy RPHash on Map Reduce(MRv2) Hadoop framework for parallel computation:}
Building the parallel RPHash for the standard open source map reduce implementation will allow us
to test the scalability performance using both internal and publically available cloud resources. In addition,
due to Hadoop and public cloud computing resources availability, this goal will serve as a roadmap for
other researchers use of the developed system.

  \item \textbf{Compare the Accuracy, Speedup, and Scalability of RPHash:} The availability of
  state of the art data clustering algorithms for Hadoop, provided by the Mahout Data Mining library,
 will allow us to compare the accuracy, performance and scalability of our system against a variety
 of optimized alternatives. Comparisons will be made based on overall clustering accuracy, such as
precision recall, scaled speed, and overall problem size scalability.
 \item \textbf{Show RPHashes innate resistance to de-anonymization attacks:} Though proving security
without rigorous proof is an open problem, we hope to show empirically the resistance of RPHash to
various de-anonymization attacks and data scraping. This rudimentary analysis, wil focus on
understanding the amount of destructiveness the random projection method has on data and its resistance
to least squares inversions.
\end{enumerate}


\subsection{Impact}

In this proposal we will develop a a novel algorithm for scalable data clustering in
the cloud. The algorithm developed combines a variety of approximate and
randomized methods for providing solutions to real world data problems. Our
focus on a non-deterministic clustering algorithm is somewhat uncommon in
computing, but coincides well with the ill-posed problem of clustering. This
follows naturally from a similar inversion regarding clustering and computing,
in which real world problems tend to converge faster than theoretically posed
problems. Furthermore, the addition of noise to theoretically posed problems
often yield more favorable results than the theoretical problem alone.\\ 

Clustering is one of the most commonly used methods for the analysis of 
unlabeled data.  Fields
in health and biology are benefited by data clustering scalability. Such fields
as Micro Array clustering, Protein-Protein interaction clustering, medical 
resource decision making, medical image processing,  and clustering of
epidemiological events all serve to benefit from larger dataset sizes.\\
In this proposal we plan to create a highly scalable, cloud based data 
clustering
system for health care providers and researchers.  This system will allow
researchers to scale their clustering problems without the need for  specialized
equipment or computing resources. Our cloud processing solution will allow
researchers to arbitrarily scale their processing needs using virtually
limitless commercial processing resources. \\ 

\section{Background}
Data clustering is a useful tool in pilot data and knowledge engineering. It's uses range from practical data localization problems,
to helping shape our understanding of complex biological, social, and physical structures. Though the problem itself
has been shown to be NP-Hard\cite{dasgupta08}\cite{Mahajan09}, many algorithms exist that usually terminate in polynomial 
time with good results.
\subsection{Big Data Clustering}
In light of the ``Big Data'' revolution, many machine learning algorithms have been modified to work on
very large, distributed datasets. The goal of clustering is the same, however memory and network
constraints tend to dominate the running time requirements over algorithmic complexity alone. RPHash
is a clustering algorithm designed from the ground up to deal with these problems head on. Instead
of modifying an existing algorithm to perform better across networks and distributed systems, RPHash
trades redundant computation for lower memory and network bandwidth requirements through the use
of random projections, and universally deterministic hashes.


\subsection{Random Projection Methods}
The resurgence of the random projection method of Johnson and Lindenstrauss was
reinvigorated with the work of Achlioptas on Database Friendly Projections\cite{Achlioptas01}, that provided
good subspace embeddings requiring minimal computation cost.
\begin{Theorem}[JL - Lemma]
 $$
(1-\epsilon) \|u-u'\|^2 \leq \|f(u)-f(u')\|^2 \leq (1+\epsilon) \| u-u' \|^2
$$
\end{Theorem}
Johnson-Lindenstrauss lemma gives a tight bound for discretion of $n$ vectors
subjected to random projections in $\mathbb{R}^d$, $\Theta( {{log(n)}\over
  {\varepsilon^2 log(1/\varepsilon) }})$.
Vampala gives a relaxation of the JL-bound of $d = \Omega(n)$ \cite{Vempala}, with a scaling factor of ${{1}\over{\sqrt{n}}}$
to preserve approximate distances between projected vectors.

An additional benefit of Random Projection for mean clustering, is that randomly projected asymmetric clusters
tend to be spherical in lower dimensional subspace representations\cite{bingham}.

\subsection{Discrete Subspace Quantizers}
From the continuous spaces of random projections we switch to discrete space partitions lattices. 
\begin{Definition}[Locality Sensitive Hash Function]
let $\mathbb{H}=\{h:S \rightarrow U\}$ is $(r_1,r_2,p_1,p_2)-$sensitive if for any $u,v\in S$
 \begin{enumerate}
   \item if $d(u,v) \leq r_1$ then $Pr_{\mathbb{H}}[h(u)=h(v)]\geq p_1$
   \item if $d(u,v) > r_2$ then $Pr_{\mathbb{H}}[h(u)=h(v)]\leq p_2$
 \end{enumerate}
\end{Definition}
%Though the
%combination of Random Projection and Space Partitioning is not new \cite{Indyk} its use in parallel computation
%is.
%ball trees, motifs, lattices\\
%signal theory, Shannon Limit\\
\subsection{Issues with random projection and high dimensions}
The \textbf{Curse of Dimensionality} 
Consider the distance between any two points
$x$ and $y$ in $\mathbb{R}^d$. under Euclidean distance: $dist(x,y)
=\sqrt{\sum^{d}_{i=1}{(x_i-y_i)^{2}}}$.  Now if we consider the vector as being
composed of values that have a uniformly distribution, according to Ullman
\cite{Ullman} the range of distances are all very near the average distance
between points.  Given this, it becomes difficult to differentiate between
near and far points, or in other words, the distance function looses its usefulness
for high dimensions\cite{Beyer}. 
The usefulness of this principle metric in clustering suggests
that approximate and exact algorithms will have an intersection at some number of dimensions.
Though this intersection is not guaranteed to be at a point of favorable clustering accuracy,
Vempala suggests random projection actually may solve some of these issues\cite{Vempala}\cite{bingham}.\\
The \textbf{Occultation Problem} arises when two 
discrete clusters in d-dimensional space project in to
two non-discrete clusters in a lower dimensional space. 
Our use of 24 dimensions helps mitigate this problem. 
The probability 
occultation for d-dimensional space to 1-dimensional space is given in 
\cite{Urruty2007}. Repeated projections result in a probability of occultation for
every projection converging to 0 as the number of projections increases. In the
case of 24 dimensional projection, we achieve a slightly faster convergence
 as the 24-dimension random subspace's vectors are nearly 
orthogonal for Gaussian distribution\cite{Vempala}. Furthermore 
the convergence from Urruty: $\underset {d\rightarrow \infty}{lim}1-\big({{2(r_1+r_2)}\over {\pi \|d-c\|}}\big)^d$
is exponential in $d$.


\subsection{Mapreduce, Hadoop, and Cloud}
\textbf{Mapreduce} is a framework for data parallel processing using two functional programming mainstays, the map and reduce
function, In addition to providing a standard parallel algorithmic structure, mapreduce also performs many of the standard
resource management tasks required for parallel processing. \\
\textbf{Hadoop} is a freely distributed, open source implementation of the mapreduce framework. Currently in its second version
utilizing the new Yarn resource manager, Hadoop will provide the networking tasks required for RPHash.\\
\textbf{Cloud} is a somewhat nebulous term as it is often used to describe anything involving web services. In regards to this
proposal, cloud will always refer to externally provided computing resources, and in particular Amazon's implementation of
Hadoop on their EC2 web service.


\section{Proposed Research}
Clustering algorithms offer
insight in a multitude of data analysis problems.  Clustering algorithm are
often limited in
their scalability due to communication requirement bottlenecks. 
 A variety of methods have been  proposed to combat this issue with varying
degrees of success. In general, these solutions tend to be parallel patches and
modification of existing clustering algorithms. \emph{RPHash} is a clustering 
algorithm designed uniquely for low interprocess communication 
distributed architectures. 
Random projection clustering 
has been previously explored in \cite{Urruty2007}, \cite{florescu09}, 
\cite{fernrandom}, \cite{avogadri09}, on single processing architectures. 

Despite theoretical results showing that k-means has an
exponential worse  case complexity \cite{Vattani}, many real world problems tend
to fair much  better under k-means and other similar algorithms.  For this
reason, clustering  massive datasets, though suffering from unbounded complexity
guarantees, is still a very active area in computing research.   Approximate and
randomized methods are common  tools for overcoming complexity growth.  In our
algorithm, called Random Projection Hash  (\emph{RPHash}) algorithm, we
utilize both approximate and randomized techniques to provide a scalable,
approximate parallel system for massive dataset clustering. 
RP-Hash performs multi-probe, random
projection of high dimensional vectors to be hashed into the unique subset
boundaries of the Leech Lattice($\Lambda_{24}$). 

\subsection{Objective 1: Develop the  Sequential RPHash Algorithm}
\subsubsection{sequential algorithm}

log-near searches around buckets\cite{panigrahy}\\
explanation and complexity analysis\\

sequentially its a dog\\




An outline of the
steps in
\emph{RPHash} is given below, however we would also like to highlight some
aspects of its function in regards to randomness and approximation. One way in
which the \emph{RPHash} algorithm achieves scalability is through the generative
nature of
its region assignment.  Clustering region assignments are performed by decoding
vector points into partitions of the Leech Lattice.  The Leech Lattice is a
unique lattice that provides optimal sphere packing density among 24 dimensional
regular lattices\cite{Cohn}.  
\begin{figure}%{l}{0.44\textwidth}
\centering
\includegraphics[scale=0.48]{doc/projplane.pdf}
\caption{Random projection on a plane\label{projplane}}
\end{figure}
Though optimal, due to the curse of
dimensionality, the overall density is somewhat sparse, requiring that the
algorithm apply shifts and rotations to the lattice to fully cover the
$\mathbb{R}^{24}$ subspace. Furthermore, in general vectors will be greater than
24 dimensions.  We cite the Johnson-Lindenstrauss (\emph{JL}) lemma to provide a
solution to
this problem (Figure \ref{projplane}).  \emph{JL} states that for an arbitrary
set of n
points in m dimensional space, a projection exists onto a d-dimensional subspace
such that all points are linearly separable with $\epsilon$-distortion following
$d \propto {\Omega({{ log(n) }\over {\epsilon^2 log 1/\epsilon}  })}$.  Though
many options for projections exists, a simple and sufficient method for high
dimensions
 is to form the projection matrix $r_{ij}\in\textbf{R}$ is $m\times d$ as
follows.
%\begin{wrapfigure}{l}{0.41\textwidth}
\[
    r_{ij}= 
\begin{cases}
    +1, & \text{with probability } {1 \over 6}\\
     0, & \text{with probability } {2 \over 3}\\
    -1, & \text{with probability } {1 \over 6}\\
\end{cases}
\]
%\caption*{ Approximate Random Projection Matrix\cite{Bingham}}
%\end{wrapfigure}

\begin{figure}%{r}{0.55\textwidth}
\centering
\includegraphics[scale=0.55]{doc/randproj.pdf}
\caption*{Multiple Projection Region Assignments and
Cardinalities}\label{randproj}
\end{figure}
\textbf{Overview of the sequential algorithm}
The basic intuition of \emph{RPHash} is to combine 
random projection with discrete space quantization,
and regard regions of the k-highest density as centroid candidates.
 According to
\emph{JL} lemma, the sub-projections will conserve the pairwise distances
in the projected space for points with $\epsilon$-distortion.
In addition to compressing a dataset to a computationally more feasible
subspace, random 
projection can also make \emph{eccentric} cluster more
spherical\cite{Dasgupta2000}\cite{Vempala}.\\
The next piece of \emph{RPHash} are discrete space quantizers. For our
implementation
of \emph{RPHash} we will rely on the Leech lattice as our region quantizer. The
Leech
lattice unique lattice in 24 dimensions that is the densest lattice packing of
hyperspheres
in 24 dimensional space. The Leech lattice is an optimal regular lattice space quantizer.
The 24
dimensional subspace partitioned by the Leech Lattice is small enough to exhibit the
spherical clustering benefit of random 
projection. Low distortion random embeddings are also feasible for very large datasets
($n = \Theta(c^{24})$) objects while avoiding the occultation phenomena\cite{Urruty2007}.
 

Furthermore, the decoding of the Leech
lattice is a well studied subject, with a constant worse case decoding complexity 
of 331 operations\cite{Vardy95}.
\\
Space quantizers have hard margin boundaries and will only correctly decode
points that are within
the error correcting radius of its partitions. This is an an issue found in
approximate nearest neighbor search \cite{panigrahy}\cite{Andoni}, and we choose
to
overcome it in a similar way as in Panigrahy\cite{panigrahy} by performing multiple
random projections of a vector then applying the appropriate locality sensitive to provide a set of
hash IDs. Using many random projections
of a vector
allows the high dimensional vector to be represented as
'fuzzy' regions probabilistically dependent on the higher dimensional
counterpart. From Panigrahy\cite{panigrahy}, we are given a bound of ($\Theta(log(n))$) for
 the required number of random projection probes to achieve 
hash collisions for a bounded radius, r -near vectors.
Figure \ref{randproj} shows 
an example of this process as a set of probabilistic collision regions.
Random projection probing adds a $\Theta(log(n))$ -complexity coefficient to the clustering algorithm
but is essential for compute node independence as will be addressed in the parallel extension of this algorithm. 
The intuition of this step is to find lattice hash IDs that on average generate more
collisions, than others. The the top $k$ cardinality set of lattice hash ID vector subsets 
represent regions of high density. The centroids are computed from the means of
the non-overlapping subsets of vectors 
for each high cardinality lattice hash ID.

\begin{figure}
        \centering
        \begin{subfigure}[b]{0.49\textwidth}
                \includegraphics[width=\textwidth]{doc/TimevaryPart}
                %\caption{Varying Vectors}
                \label{TimePart}
        \end{subfigure}
        \begin{subfigure}[b]{0.49\textwidth}
                \includegraphics[width=\textwidth]{doc/TimevaryDim}
                %\caption{Varying Dim}
                \label{TimeDim}
        \end{subfigure}
	  \caption{Computation Time for K-Means(green) and RPHash(blue) varying Vectors and Dimensions}\label{timecomplex}
\end{figure}

\begin{figure}
        \centering
        \begin{subfigure}[b]{0.49\textwidth}
                \includegraphics[width=\textwidth]{doc/PRvaryClusters}
                %\caption{Varying Clusters}
                \label{PRaccClu}
        \end{subfigure}
        \begin{subfigure}[b]{0.49\textwidth}
                \includegraphics[width=\textwidth]{doc/PRvaryDim}
                %\caption{Varying Dimensions}
                \label{PRaccDim}
        \end{subfigure}
	 \caption{Precision-Recall for K-Means(green) and RPHash(blue) varying Vectors and Clusters}\label{praccuracy}
\end{figure}


The algorithm used above has a natural extension to a parallel algorithm
which we will describe in the following section.\\

\subsection{Objective 2:  Develop A distributed version of the proposed RPHash Algorithm}
A key aspect of generative groups such as the IDs of lattice centers in the Leech Lattice, is that their 
values are universally computable. For distributed data systems this aspect provides complete
compute node independence. Furthermore,  r-near neighbor hash collisions for projected vectors
do not need to share a particular basis. Instead they need only be randomly projected a bounded number of
times according to results found in Panigrahy\cite{panigrahy}.
Using these tools, the \emph{RPHash} algorithm
is mostly naively parallel. The overall parallel portion of the algorithm is in fact no
more complicated than a
parallel sum algorithm, or reduce operation on Hadoop. 
%Parallel prefix sum which has a work efficient parallel implementation of Ladner requiring $\Theta(n)$-steps 
%and $\Theta(log(c))$ communications for $c$ computing nodes\cite{Ladner}. 

Algorithmic adaptations to parallel sum problem is yet another common tool for attaining
parallel scalability and limiting complexity growth.
Due to the work efficient parallel complexity of the reduce function, the overall complexity of RPHash is
asymptotically no worse than its sequential complexity. 
This complexity, as previously stated is dominated by the number of required 
random projection hash probes for $n$ vectors($\Theta(nlog(n))$). 
The data transfer requirement of the
directly applied parallel sum for
\emph{RPHash} is on the order of the dimensionality of the vectors and the
parameter $k$ for each compute
node. To minimize this, with no overall effect on the complexity order of
Parallel \emph{RPHash}, we
propose a two step approach. The steps will be summarized here, as well as in
standard psuedo-code
algorithmic form (\ref{Phase1} and \ref{Phase2} respectively).\\
%To achieve the previous goals, we alter the
%standard mean shift kernel gradient  descent algorithm to common density modes
%by separating it into a two step  course-grain and fine-grain shift.  By
%adjusting the granularity of the shift,  it is thought that we can control the
%compute node inter-dependency.   Furthermore, one goal of this paper is to be
%able to draw a correlation  between the $\epsilon$-error of our algorithm to
%optimal clustering, and the  course-grain to fine-grain ratio parameter.\\
%\begin{wrapfigure}{r}{.5\textwidth}
%\centering
%\[
%\overset{\rightarrow}{x}= {{\sum_{x_i\in N_(x)}
%{K'({x-x_i})\overset{\rightarrow}{x}}}\over {\sum_{x_i\in N_(x)} {K'({x-x_i)}}
%}}
%\]\caption*{Mean-Shift Iteration\cite{Fajie}}
%\end{wrapfigure}
%\textbf{Fine-Grain Iteration:}\\
\textbf{Phase 1 of Parallel RPHash: Maximum Bucket Counting.}\\
As \emph{RPHash} is a distributed algorithm, we will assume the data 
vectors reside on $D$ independent computing nodes, connected through
any non-specific, Hadoop\cite{hadoop} compatible network hardware. The system
architecture
will utilize the Yarn \emph{MR}v2 resource manager for task scheduling
and network communication.  With much of the parallel task scheduling
details handled by Yarn  \emph{MR}v2, we focus our algorithm description on the
per compute node task level.  
As the random projections can be performed independently, the compute 
node interdependency is very low.
Due to the Leech lattice's set
dimensionality, conversion between vectors of $\mathbb{R}^d$ and 
$\mathbb{R}^{24}$  will be performed by a random projection matrix of random
variables
following a nearly Gaussian distribution as in Bingham\cite{bingham}.
Therefor our first parallel task is to distribute a random projection matrix to
each compute node. 
To further minimize communication load, we will note the fact that the random
projection matrix, though, stated as random, is actually the result of a
pseudo-random
number generation sequence that is uniquely defined by a single seed number.  A
trade-off
exists between the communication cost of distributing a large
$\mathbb{P}_{m\rightarrow d}$
and the redundant calculation of such data per compute node. Parallel processing
tuning 
is an effective solution to this problem, however in \emph{RPHash} we choose the
generative
method, due to the constant scalability barrier that results from network
bandwidth saturation
and empirical results from Chowdhury\cite{chowdhury} suggesting that the ratio
of communication
to computation increases with the number of MR nodes. 
With the initial data distribution requirements taken care of, we can return to
focusing on 
per node computation. Each compute node processes its set of vectors
independently; first 
projecting it to a random subspace, then decoding to its nearest Leech lattice
centroid ID.
The lattice region IDs are generative, and are thus identical across all compute
nodes.
In the `bucket
counting' phase, we are only concerned with the number of collisions each
lattice ID receives.
Once all data vectors have been processed and their collisions accumulated, 
we will use the standard parallel prefix sum algorithm of 
Ladner\cite{Ladner} to accumulate the lattice ID counts across all compute
nodes. The set of
lattice ID counts is sorted, and the $k\log(n)$-largest Leech lattice IDs are
broadcast to all compute
nodes for Phase 2 processing.
%The lattice region IDs are generative,  and can be assigned either by
%truncating
%the ID mod the number of available  compute nodes, or some more advanced load
%balancing technique based on the  number of vectors per compute node.\\ 
%Using techniques from Andoni \cite{Andoni} and an addition from
%Carraher\cite{Carraher}, computation of the function $N(\cdot)$ of near
%neighbors for a given point can be performed in nearly optimal
$\Theta(n^\rho)$-time.  
For a Leech lattice hash function where $d \leq24$,
$\rho \approx 0.2671 $\cite{Andoni}, while linear search requires $\Theta(d
n)$-operations.  This gives an internal clustering complexity of $\Theta(k
n^{1+\rho} )$ where $k$ is the number of vectors per compute node.  


Note: This may result in lost clusters, however preliminary results show our algorithm performs
as good as or better than K-means for the same dataset, in dimensions greater than $d<100$.

Show graph of precision recall here.

%By adding the random
%projection aspects to this iteration, it is believed we will achieve some
%tempering to the error surface across multiple projections  which may help
%avoid
%some local minima of the density modes by smoothing out  point
%discontinuities. 
%Furthermore, the sigma value for an Gaussian, $N(\mu,\sigma)$ distribution in
%the random projection may allow us to control tempering.
%\begin{wrapfigure}{r}{0.52\textwidth}
%

\begin{algorithm}
\caption{Phase1 RP-Hash Clustering}\label{Phase1}
\begin{algorithmic}[1]
\REQUIRE $X=\{x_1,...,x_n\}$, $x_k \in \mathbb{R}^m$ - data vectors\\
 $D$ - set of available compute nodes\\
$\mathbb{H}$ - is a d dimensional LSH function\\
 $\widetilde{X} \subseteq X$ - vectors per compute node\\
 $\mathbb{P}_{m\rightarrow d}$ - Gaussian projection matrix
\STATE $C_s=\{\varnothing\}$ - set of bucket collision counts
\FORALL{$x_k \in \widetilde{X}$} 
\STATE $\tilde{x_k}\leftarrow \sqrt{{m}\over{d}}\mathbb{P}^{\intercal}x_k $
\STATE $t = \mathbb{H}(\tilde{x_k})$
\STATE $C_s[t]=C_s[t]+1$
\ENDFOR
\STATE sort($\{C_s,C_s.index\}$)
\State \textbf{return}$\{C_s,C_s$.index$\} [0 : k\log(n)]$
%\caption*{Phase1 RP-Hash Clustering}
\end{algorithmic}
\end{algorithm}
%\end{wrapfigure}




\textbf{Phase 2 of Parallel RPHash: }
The two phased approach for clustering is motivated by the work of Panigrahy,
Andoni, and Indyk \cite{panigrahy}\cite{Andoni}
in c-approximate nearest neighbor search(c-ANN). In the first phase,  the search
database of high cardinality lattice hashes is constructed/
While actual searching of the database is performed in the second phase.
\emph{RPHash} is a slight inversion of c-ANN, in the
second phase, all vectors of the database are searched, against the small subset
of high cardinality lattice hash ID sets.
The set of lattice hash ID's corresponding to the $k*log(n)$ largest lattice
hash ID subsets are broadcast to all compute nodes through standard MapReduce 
broadcast methods. In the case of overlapping lattice hash set, clusters having
similar centroids, an overlap factor of $log(n)$ is applied to the parameter
$k$, so overlapping subsets can be merged during the gather phase with little
effect on computational complexity.\\
The per compute node processing of Phase 2 requires that we perform the
projection and hashing process from Phase 1 for $log(n)$ random projections.
While processing the projection and hashes, we must also store the sums of the
original vectors for each $klog(n)$ cluster IDs. After processing the
$log(n)$-projections of n
vectors, we use a similar parallel prefix scan method for vectors, to sum up all
the vectors and counts. The means for each vector sum
are then computed using the set of lattice ID collision totals.

%\begin{wrapfigure}{l}{0.55\textwidth}
%
\begin{algorithm}
\caption{Phase2 RP-Hash Clustering}\label{Phase2}
\begin{algorithmic}[1]
\REQUIRE $X=\{x_1,...,x_n\}$, $x_k \in \mathbb{R}^m$ - data vectors\\
$D$ - set of available compute nodes\\
$\{C_s,C_s.$index$\}$ - set of $klogn$ cluster IDs and counts\\
$\mathbb{H}$ - is a $d$-dimensional LSH function\\
$\widetilde{X} \subseteq X$ - vectors per compute node\\
$p_{m\rightarrow d}\in \mathbb{P}$ - Gaussian projection matrices
\STATE $C=\{\varnothing\}$ - set of centroids
\FORALL{$x_k \in \widetilde{X}$} 
\FORALL{$p_{m\rightarrow d} \in \widetilde{\mathbb{P}}$} 
\STATE $\tilde{x_k}\leftarrow \sqrt{{m}\over{d}}p^{\intercal}x_k $
\STATE $t = \mathbb{H}(\tilde{x_k})$
\IF{$t \in C_s.$index}
\STATE $C[t]=C[t]+x_k$
\ENDIF
\ENDFOR
\ENDFOR
\State \textbf{return} ${C}$
%\caption*{Phase2 RP-Hash Clustering}
\end{algorithmic}
\end{algorithm}
%\end{wrapfigure}
\subsubsection{Drawbacks}
For a fixed number of random projections the probablility of finding a goof projections
converges exponentially to zero as the number of dimensions $d$ increases.\cite{florescu09}

\subsection{Objective 3: Deploy RPHash to EC2 Cloud's Hadoop Implementation}
An important goal of RPHash and big data mining for the computing community in general
is accessability to data scientists. RPHash is not focused on exotic, difficult to
access systems, and instead is intended to be built for cloud processing platforms.\\

The first requirement of deploying our parallel algorithm on a cloud based platform is
to create a simulated environment on a local system. Virtualization will be used to
create this simulated environment, and scripts(Chef, Puppet) will be written to handle the automatic
deployment of virtual machines running the Hadoop MRv2 framework. Automatic deployment
scripts will help with speedup tests in objective 4 both locally and on the EC2 Cloud.\\

After a local deployment system is available, an Amazon EC2 cloud account will be created for
further testing. The amazon account will be managed by the researchers using funds from
this grant for Amazon's pay-per use computing services. Most EC2 testing will be 
performed around scalability testing on synthetic data. This requirement will avoid
the expenses associated with Amazon's services, while also not violtating potential
copyright and privacy issues that could arise from uploading to the cloud.

\subsection{Objective 4: Demonstrate Accuracy, Speedup and Scalability of RPHash}
As RPHash is a new method for clustering, its accuracy will have to be tested on
a variety of types of data. Though specifically intended for high dimensional
data, various synthetic, real, and combination datasets will be tested. Additionally
RPHash will be compared with the current state-of-the-art parallel clustering
algorithms.\\
Mahout is our primary resource for such algoriths, as they have been openly designed
and extensively optimized for running on the Hadoop MRv2 engine. RPHash will be
tested agains these systems in three areas of algorithmic interest.
\begin{itemize}
 \item Accuracy - a 50:50 model training set will be created from the three types of
 datasets. Common accuracy measures will be taken such as the common Precision-Recall
 test, ROC test, and TF/FP tests. Though a variety of other tests exist for clustering
 accuracy, we mostly expect to approach the other algorithms of Mahout in accuracy, as
 our real goals are focused on the latter tests.
 \item Speedup - is simply a measure of how well we utilize the addition of processors. Unfortunately
 speedup is very commonly subject to Amhdals law, and the return on investment often falls of quickly.
 Though this is an almost unavoidable constraint on parallel algorithms, RPHash should perform very well
 as its sequential requirements are fairly small.
 \item Scalability - Another important test of any clustering algorithm is how well it scales
 with increasing dataset sizes. Though we have rough estimates from sequential tests and theoretical
 algorithmic design on how RPHash should scale, true empirical results will offer a more robust
 method of comparing with the algorithms of Mahout.
\end{itemize}



\subsection{Objective 5: Data Security}
Though certain privacies are provided
by cloud service providers, the security itself is entirely dependent upon the often, private
companies security practices. For certain problems, specifically those involving private individuals 
healthcare data, such privacy safegaurds must undergo rigorous auditing requirements. Such 
auditing is out of the scope of a private cloud companies business, making the move to cloud
resources difficult for the health care community.
Furthermore, recent United States government initiatives pushing for the large scale
availability
of data resources have made vast quantities of de-identified health information
available
to the public. These resources however have prompted advances in attacks on de-identification of 
whole genome sequence data. Such attacks have been used to
to associate, thought to be, anonymous medical records with specific individuals\cite{deident}.
Similar de-anonymization 
attacks \cite{deanon1}\cite{deanon2} along with a presidential commission
(privacy and progress in WGS)
have prompted a need for better data security of medical records data. Our
algorithm provides an
intrinsic solution to this problem in both the distribution of data among
servers as well as during the communication steps required by the algorithm.

\emph{RPHash} is a distributed algorithm designed to compute clusters on 
databases that could potentially span over the public internet.
 As such, data it uses to
compute clusters can theoretically reside independently among different health
care facilities with no requirement
for any single location data storage architecture. 
As a consequence of projecting the real data vectors to a random subspace via a
near, but not completely orthogonal
matrix, destructive data loss occurs providing a cryptographic ''trap-door'
function. The data loss is an intrinsic
part of the \emph{RPHash} clustering algorithm that has little adverse effect on its
model generation and subsequent recall accuracy.  
\\
The only point at which fully qualified vectors are transmitted between compute
nodes is during the Phase2 gather stage. This however is not of concern as
the data represents an aggregation of individual data. By definition of a centroid it is the
average of only the $k$ largest populations of patients.
\emph{RPHash} is a clustering algorithm for very large datasets, where the
probability of identifying an individual from Phase2 vector
 information is proportional to the size of the smallest identified cluster.

\section{Experimental Approach}


The experirimental approach for testing our algorithm will consider 4 major attributes for 
parallel maching learning algorithms.

\subsection{Algorithm Accuracy}
A classic analysis of any clustering algorithm is its ability to correctly catagorize unseen data. There are various ways
of doing this such as Precision Recall, ROC, and FP/TF with varying applicability to different domain constraints. We
plan to provide this data for our algorithm along with various other algorithms performed on large datasets, and in 
parallel using the MRv2 framework, and distributed systems. A primary resource for comparison will be the industry 
standard learning algorithms found in the Apache Mahout Machine learning for Hadoop Project. We will use these
tools to evaluate both the classification accuracy between RPHash in win/draw/lose contest, as well as the overall
speed and scaling complexity of the various algorithms.

\subsection{Data}
To accurately test our system, a variety of real and synthetic datasets will be incorporated. Our synthetic datasets will
consist of varying size and dimension, gaussian clusters, while our real datasets will be acquired from various bioinformatic
databases, as well as semantic data from health care resources. An emphasis on health care related data will
be used to establish the distributed database security utility of our algorithm. \\
Our synthetic dataset will consist of $k$ vectors from $\mathbb{R}^d$ choosen from a uniform random distribution.
These vectors will form the basis for our cluster centers. For each cluster center an $n/k$ set of vectors
will be generated by an additive gaussian noise permutation added to the center vector. Ultimately
this will give us $n$ vectors in $k$ Gaussian clusters. From this we will reserve half of the vectors for
training, and the other half for testing accuracy.\\
Our real dataset will consist of genomic data and image SIFT vector data. These sets will give us
a respective demonstration of RPHash's performance on very high dimensional, and relatively high dimensional 
datasets.


\subsection{Security Tests}
As our main goal is to show RPHash's resistance to de-anonymization attacks by use of destructive
projection functions, the primary security demonstration will be to show that no single fully qualified
record can be associated with the data being sent between Hadoop Nodes.\\
Auditing security is a difficult task, and is somewhat beyond the scope of this proposal,
as such, security demonstrations will be primarily empirical. Known correlated fully qualified
vectors and projected vectors will be compared for similarity using the least squares solution
of the projection matrix. For orthogonal projection matrices the least squares solution serves 
as a suitable inverse for the projection function. However, for non-orthogonal projection matrices
the least-squares solution is over-determined, and the basises overlap, causing unrecoverable
data loss. Our goal in testing is to show that this data loss is enough to make it 
very difficult to re-associate vectors with any projected data shared among Hadoop Nodes.\\ 

\section{Related Work}
Clustering is a fundemental machine learning algorithm and therefor has been 
extensively applied to a variety of dataset sizes. Furthermore, due to the inherent
scalability issues of many sequentially designed algorithms, parallel
algorithms have also been developed to mitigate these issues.
\subsection{Density Scan Database Clustering Methods}
The first set of parallel clustering algorithms began with density based scanning 
methods. These methods tend to work well on spacially seperated datasets with relatively
low dimension. A very common clustering problem for these types of algorithms
would be on geospitial data, such as large maps and image segmentation. The algorithms
DBScan\cite{dbscan}, Clique\cite{clique}, and CLARANS\cite{clarans}, respectively, represent a 
successful progression of the density scanning techniques.
Though density scan algorithms are an example of parallel designed algorithms
like RPHash, they often show weaknesses in accuracy when scaling the number of dimensions. 
A proposed solution mentioned below to this problem is PROCLUS\cite{proclus}.

\subsection{Random Projection}
Another important feature of RPHash is the use of random projection clustering.
One of the first algorithms to explore random projection for clustering was Proclus\cite{proclus}.
Proclus used the assumptition that high dimensional data often tends to be overly sparse
and feature subset selection would offer a way of compacting the problem. Many
subset selection algorithms have been explored\cite{subset1}\cite{subset2}, they often
require computationally expensive gradient methods. This idea, combined with the near
accuracy of random projections in preserving discrepancy. From their, the authors 
suggest an iterative method of medoid searching similar to CLARANS\cite{clarans}.
Though this method is successful in accelerating somewhat larger dimensionality problems ($~d=20$),
it is still relatively sequentially bound in its iterative nature.\\

In addition to PROCLUS, various other methods and analysis have been performed
on clustering with random projections that provide some bounds on the convergence
and extensibility of random projection clustering. Florescu gives bounds on the
scaling and convergence of projected clustering\cite{florescu09}. Their results
closely follow the logic of Urruty\cite{Urruty2007}, and find that the number
of trial projections, or similarly in our case, projected dimensions, effects
the convergence rate exponentially. They also assert the probability 
of random projection offering a good partitioning plane decreases exponentially
as the number of dimensions increases. This suggests RPHash may have to find a
way to overcome this issue, or show empirically that the assymptotic behaviour
of increasing projected dimensions, overshadows increases in $d$.\\

Other clustering by random projection algorithms have been explored that are very similar to
our proposed method, but for projections on the line. These so called cluster ensemble approaches
\cite{fernrandom}\cite{alweighted06}\cite{avogadri09} use histograms to identify 
cluster regions of high density much like RPHash. Though, as suggested in Florescu and Urruty,
the single dimension approach may plagued by issues of occultation, and exponential convergence
as $d$ increases.



\section{Proposed Research Schedule}  
will constitute the majority of our efforts.  As it is the algorithm
behind the tests in our subsequent aims, it must be completed and tested prior
to use.  The testing phase against Mahout clustering algorithms however need not
be performed at this time. As the development and deployment  on commercial
cloud computing resources is by in large the longest duration task, we are
allotting 6-8 months to complete it.\\ 
Aim 4. Is an ongoing demonstration of the security tolerance of the system from


\bibliography{refs}
\bibliographystyle{ieeetr} \markright{ }
\end{document}
