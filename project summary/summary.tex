
\documentclass[11pt]{article}

\usepackage{fullpage,times}
\setlength{\topmargin}{0.4in}
\addtolength{\textheight}{1in}

\thispagestyle{empty}
\begin{document}

\noindent
\begin{center} {\bf Project Summary} \end{center}

\medskip
\noindent

%% Each proposal must contain a summary of the proposed project not more than one page in
%% length. The Project Summary consists of an overview, a statement on the intellectual
%% merit of the proposed activity, and a statement on the broader impacts of the proposed
%% activity.  

%% The overview includes a description of the activity that would result if the proposal
%% were funded and a statement of objectives and methods to be employed. The statement on
%% intellectual merit should describe the potential of the proposed activity to advance
%% knowledge. The statement on broader impacts should describe the potential of the
%% proposed activity to benefit society and contribute to the achievement of specific,
%% desired societal outcomes. 

%% The Project Summary should be written in the third person, informative to other persons
%% working in the same or related fields, and, insofar as possible, understandable to a
%% scientifically or technically literate lay reader. It should not be an abstract of the
%% proposal. 

%% Proposals that do not contain the Project Summary, including an overview and separate
%% statements on intellectual merit and broader impacts will not be accepted by FastLane
%% or will be returned without review. 

%% If the Project Summary contains special characters it may be uploaded as a
%% Supplementary Document. Project Summaries submitted as a PDF must be formatted with
%% separate headings for the overview, statement on the intellectual merit of the proposed
%% activity, and statement on the broader impacts of the proposed activity. Failure to
%% include these headings may result in the proposal being returned without review. 

\subsection*{Overview}

This proposal presents a distributed algorithm for secure clustering of high dimensional
data.  The novel algorithm, called Random Projection Hash or \emph{RPHash}, utilizes
aspects of locality sensitive hashing (LSH) and multi-probe random projection for
computational scalability and linear achievable gains from parallel speed up.  The two
step approach is data agnostic, minimizes communication overhead, and has a priori
predictable computational time.  The system is deployable on commercially available cloud
resources running the Hadoop (\emph{MR}v2) implementation of MapReduce.  The \emph{RPHash}
solution will have a wide applicability to a variety of standard clustering applications
while this project will focus on a subset of clustering problems in the biological data
analysis space.  \emph{RPHash} also combats de-anonymization attacks inherently resulting
from its algorithmic requirements thus addressing requirements involving the handling and
privacy protection of health care data\cite{presidential} as well as the inherent privacy 
concerns of using
cloud based services.  Furthermore, \emph{RPHash} will allow researchers to scale their
clustering problems without the need for specialized equipment or computing resources.
The proposed cloud processing solution will allow researchers to arbitrarily scale their
processing needs using virtually limitless commercial processing resources.

\subsection*{Intellectual Merit}
A principle driving force in computational progress results from material and architectural
advances in microprocessor design. Many of these advances have been stagnated for linear processing 
however due to thermal dissipation and energy requirements, resulting in a shift toward parallel multi-processing.
Though much has been done to adapt current algorithms for this parallel processing landscape, many
solutions tend to be a' posteriori methods favoring empirical speedup over long term scalability. In this
proposal, we develop a method for an algorithm central to data analysis, designed expressly for 
parallel multi-processing systems. In addition, our algorithm addresses often overlooked communication
bottlenecks in parallel design, through use of side channel synchronization based on mathematically 
generative groups and probabilistic approximation. A favorable side effect of the probabilistic
approximation results in a possible solution to de-anonymization attacks\cite{deanon1}\cite{deanon2} on user data.


\subsection*{Broader Impacts}

Clustering has long been the standard method used for the analysis of labeled and
unlabeled data.  Clusterings' effects intrinsically identify the latent models underlying
the distributions of objects
in a dataset, often unattainable through standard statistical methods.  Single pass, data
intensive statistical methods are often the primary workhorses for parallel database
processing of business logic and other domains, while clustering is often overlooked due
to scalability ocncerns and confusion caused by the wide variety of available distributed
algorithms \cite{clusters}.  A multitude of surveys \cite{clusters} have been made available
comparing different aspects of clustering algorithms, such as accuracy, complexity and
application domain.  Fields in health and biology are benefited by data clustering
scalability.  Such fields as Micro Array clustering, Protein-Protein interaction
clustering, medical resource decision making, medical image processing, and clustering of
epidemiological events all serve to benefit from larger dataset sizes. Furthermore, the proposed
method provides data anonymization through destructive manipulation of the data preventing
de-anonymization attacks beyond standard best practices database security methods.

\end{document}
 
